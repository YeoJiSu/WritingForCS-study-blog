\documentclass[11pt, a4paper]{article}
\usepackage[hangul]{kotex}
\usepackage{enumitem}
\usepackage{indentfirst}

\begin{document}

\title{인프라 관리 자동화 도구들 간의 비교 및 분석}
\author{여지수\footnote{duwltn1301@pusan.ac.kr}}
\date{2022.10.10}
\maketitle

\section{서론}
‘하나’의 네트워크 노드에 설치된 소프트웨어를 관리하기란 어렵지 않다. 
하지만, 노드가 ‘수만 개’ 일 때 이 모든 노드들에 대해 하나씩 설치 및 관리를 한다면, 반복되는 작업들을 계속해 주어야 하기에 굉장히 비효율적일 것이다. 

그렇기에 이러한 노드들을 구성하는 '인프라'를 자동적으로 관리하는 기술인 "IaC(Infrastructure as Code)"가 떠오르고 있다. 
반복적인 작업을 최소화하고 IaC를 구현하고자 하는 여러 소프트웨어 구성 관리 툴들이 개발이 되었다. 
필자는 이러한 도구들을 직접 설치해 보고, 그 특징들에 대해 분석하고 비교를 해보려 한다.

\section{관련 연구}
\subsection{코드를 통한 인프라 관리\protect\footnote{RedHat 공식 홈페이지,``코드형 인프라(IaC)란?'',
https://www.redhat.com/ko/topics/automation/what-is-infrastructure-as-code-iac}}

IT 환경을 관리하고 운영하는 데에 필요한 구성 요소가 바로 인프라이다. 
이러한 인프라를 수동적이 아닌 코드를 통해 자동적으로 관리하면 낭비되는 작업을 줄일 수 있다. 
IaC를 구현하기 위해 Chef, Puppet, Ansible, Saltstack, Terraform, AWS CloudFormation 등과 같은 툴이 있다. 
이러한 툴들을 사용하여 인프라를 보다 효율적으로 운영하고, 애플리케이션 개발에 있어서도 생산성을 높일 수 있다. 

\subsection{인프라 관리를 위한 자동화 도구들\protect\footnote{조혜영, ``시스템 설정 및 관리를 위한 자동화 관리 도구에 관한 연구', 한국과학기술정보연구원 KISTI 국가슈퍼컴퓨팅연구소, 2015.}}

PUPPET은 루비로 개발된 형상관리를 위한 도구이다. 
프로그램 설정에 필요한 데이터들은 “Module”이라는 곳에 보관이 된다. 
PUPPET은 ’서버-클라이언트’ 구조로 서버(master)에서 이 Module을 작성하고 배포하면, 클라이언트(agent)는 설치 후 서버에 보고를 하는 식의 형태를 띤다. 

CHEF는 루비와 얼랭으로 개발된 형상관리를 위한 프레임워크이다. 
실제로 프레지, 페이스북과 같은 유명한 기업들이 이 CHEF를 사용한다. 
PUPPET에서는 Module이 설정에 필요한 데이터들을 보관한다면, CHEF에서는 ‘Cookbook’이 그 역할을 맡고 있다. 
CHEF는 ‘서버-클라이언트’구조에 “workstation”이 추가된 구조이다. 
서버는 Cookbook을 보관하는 역할만 하며, workstation에서 Cookbook을 개발하고 서버에 업로드하며 배포 지시를 내린다.  
클라이언트(node라 불림)는 서버로부터 데이터를 가져와 패키지를 설치하고 애플리케이션을 실행한다. 

Puppet 과 Chef 가 기능적인 측면에서는 큰 차이가 없지만,  Chef가 Puppet보다 좀 더 유연한 코딩이 가능해서 관리자에게는 Puppet, 개발자에게는 Chef가 더 나은 도구로 보인다. 
그리고 커뮤니티 측면에서 볼 때 Puppet이 더 많은 인원들이 참여한다.

HEAT은 ‘워크플로우’를 자동화해주는 오픈스택 프로젝트 개념이다. 
그렇기에 Puppet과 Chef가 같이 쓰일 수 있다. 

\end{document}