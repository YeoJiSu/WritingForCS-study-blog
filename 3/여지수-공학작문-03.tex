\documentclass[11pt, a4paper]{article}
\usepackage{kotex} % 한글 인식 위한 패키지
\usepackage{csquotes} % 인용구 사용 위한 패키지
\usepackage{geometry}
\usepackage{indentfirst} 
\setlength\parindent{2em}
\linespread{1.3} % 줄간격 "한 줄 반"으로 설정하기
\geometry{ 
top = 0.2in,
bottom = 0.2in
}

\begin{document}

\title{AI의 핵심이 고품질 데이터라고?}
\author{여지수 \footnote{duwltn1301@pusan.ac.kr}}
\date{2022.10.03}
\maketitle

\begin{displayquote} 
\centering % 중앙 정렬
    『 AI 응용 상용화 85\% 실패, 그리고 Data-Centric AI로의 이동 』 \footnote{$https://www.samsungsds.com/kr/insights/data_centric_ai.html$}
\end{displayquote}

\section{뭣이 중헌디?}
필자는 머신러닝 개발자를 희망하는 컴퓨터공학과 학부 3학년 재학생이다. 
학교에서 머신러닝과 관련된 수업을 들을 때면 항상 이러한 궁금증을 품는다. 
'그러면 데이터의 양과 질 중 무엇이 더 중요한 것인가?' 
이에 대한 견해는 사람마다 다를 수 있지만, 필자는 그 중 데이터의 질을 더 강조하는 한 인사이트의 글에 매혹이 되었다.

\section{왜 고품질 데이터인가?}
\subsection{AI의 한계}
AI는 우리 생활의 많은 부분들에 자리를 잡고 있고 꾸준히 황금기를 이어갈 것 같지만, 사실 AI는 여러 한계들을 가지고 있다. 
기계를 학습시키기 위해서는 방대한 양의 데이터가 필요하다. 
그러나, 자연의 셀 수 없는 데이터들을 수집하는 것은 쉽지 않은 일이다.
또, 데이터들의 분포는 시간에 따라 변하기 때문에 현재로써의 학습 이론에는 한계가 존재한다. 

\subsection{해결 방안}
이러한 한계들을 해결하기 위해서는 데이터들이 방대하지 않더라도 신경망을 정확히 학습하는 이론이 필요하다. 
그렇기에 AI를 연구하는 사람들에 있어서의 숙제는 아래의 두가지가 될 것이다.
\begin{itemize}
    \item  “오염된 데이터셋을 보정하여 Clean Labels로 만드는 기술”
    \item  “품질이 좋은 새로운 데이터를 새로 자동 생성하는 기술”
\end{itemize}

\section{연구, 그리고 AI 모델 개발에도...}
결국 ‘방대한 데이터’의 수집 보다는 ‘고품질 데이터’의 생성이 더 중요하다는 것이 이 글의 주장이다. 
이는 AI 기술의 연구 뿐만 아니라 모델을 개발하는 때에도 중요하게 작용하는 부분이라 필자는 생각한다. 
모델에 학습시키고 훈련시킬 데이터들을 수집할 때에 무조건적인 방대한 데이터를 수집하기 보다는 수집 후 활용 가능한 데이터를 ‘고품질화'하는 것에 많은 힘을 쏟아 개발을 해야한다. 
그렇게 된다면 현 AI의 한계를 극복하고 정확도가 높은 모델을 개발 할 수 있을 것이다.
\end{document}
